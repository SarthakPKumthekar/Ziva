% Generated by Sphinx.
\def\sphinxdocclass{report}
\documentclass[letterpaper,10pt,english]{sphinxmanual}
\usepackage[utf8]{inputenc}
\DeclareUnicodeCharacter{00A0}{\nobreakspace}
\usepackage{cmap}
\usepackage[T1]{fontenc}
\usepackage{babel}
\usepackage{times}
\usepackage[Bjarne]{fncychap}
\usepackage{longtable}
\usepackage{sphinx}
\usepackage{multirow}

\addto\captionsenglish{\renewcommand{\figurename}{Fig. }}
\addto\captionsenglish{\renewcommand{\tablename}{Table }}
\floatname{literal-block}{Listing }



\title{zBuilder Documentation}
\date{February 24, 2017}
\release{0.9.1}
\author{Lonnie Kraatz}
\newcommand{\sphinxlogo}{}
\renewcommand{\releasename}{Release}
\makeindex

\makeatletter
\def\PYG@reset{\let\PYG@it=\relax \let\PYG@bf=\relax%
    \let\PYG@ul=\relax \let\PYG@tc=\relax%
    \let\PYG@bc=\relax \let\PYG@ff=\relax}
\def\PYG@tok#1{\csname PYG@tok@#1\endcsname}
\def\PYG@toks#1+{\ifx\relax#1\empty\else%
    \PYG@tok{#1}\expandafter\PYG@toks\fi}
\def\PYG@do#1{\PYG@bc{\PYG@tc{\PYG@ul{%
    \PYG@it{\PYG@bf{\PYG@ff{#1}}}}}}}
\def\PYG#1#2{\PYG@reset\PYG@toks#1+\relax+\PYG@do{#2}}

\expandafter\def\csname PYG@tok@gd\endcsname{\def\PYG@tc##1{\textcolor[rgb]{0.63,0.00,0.00}{##1}}}
\expandafter\def\csname PYG@tok@gu\endcsname{\let\PYG@bf=\textbf\def\PYG@tc##1{\textcolor[rgb]{0.50,0.00,0.50}{##1}}}
\expandafter\def\csname PYG@tok@gt\endcsname{\def\PYG@tc##1{\textcolor[rgb]{0.00,0.27,0.87}{##1}}}
\expandafter\def\csname PYG@tok@gs\endcsname{\let\PYG@bf=\textbf}
\expandafter\def\csname PYG@tok@gr\endcsname{\def\PYG@tc##1{\textcolor[rgb]{1.00,0.00,0.00}{##1}}}
\expandafter\def\csname PYG@tok@cm\endcsname{\let\PYG@it=\textit\def\PYG@tc##1{\textcolor[rgb]{0.25,0.50,0.56}{##1}}}
\expandafter\def\csname PYG@tok@vg\endcsname{\def\PYG@tc##1{\textcolor[rgb]{0.73,0.38,0.84}{##1}}}
\expandafter\def\csname PYG@tok@vi\endcsname{\def\PYG@tc##1{\textcolor[rgb]{0.73,0.38,0.84}{##1}}}
\expandafter\def\csname PYG@tok@mh\endcsname{\def\PYG@tc##1{\textcolor[rgb]{0.13,0.50,0.31}{##1}}}
\expandafter\def\csname PYG@tok@cs\endcsname{\def\PYG@tc##1{\textcolor[rgb]{0.25,0.50,0.56}{##1}}\def\PYG@bc##1{\setlength{\fboxsep}{0pt}\colorbox[rgb]{1.00,0.94,0.94}{\strut ##1}}}
\expandafter\def\csname PYG@tok@ge\endcsname{\let\PYG@it=\textit}
\expandafter\def\csname PYG@tok@vc\endcsname{\def\PYG@tc##1{\textcolor[rgb]{0.73,0.38,0.84}{##1}}}
\expandafter\def\csname PYG@tok@il\endcsname{\def\PYG@tc##1{\textcolor[rgb]{0.13,0.50,0.31}{##1}}}
\expandafter\def\csname PYG@tok@go\endcsname{\def\PYG@tc##1{\textcolor[rgb]{0.20,0.20,0.20}{##1}}}
\expandafter\def\csname PYG@tok@cp\endcsname{\def\PYG@tc##1{\textcolor[rgb]{0.00,0.44,0.13}{##1}}}
\expandafter\def\csname PYG@tok@gi\endcsname{\def\PYG@tc##1{\textcolor[rgb]{0.00,0.63,0.00}{##1}}}
\expandafter\def\csname PYG@tok@gh\endcsname{\let\PYG@bf=\textbf\def\PYG@tc##1{\textcolor[rgb]{0.00,0.00,0.50}{##1}}}
\expandafter\def\csname PYG@tok@ni\endcsname{\let\PYG@bf=\textbf\def\PYG@tc##1{\textcolor[rgb]{0.84,0.33,0.22}{##1}}}
\expandafter\def\csname PYG@tok@nl\endcsname{\let\PYG@bf=\textbf\def\PYG@tc##1{\textcolor[rgb]{0.00,0.13,0.44}{##1}}}
\expandafter\def\csname PYG@tok@nn\endcsname{\let\PYG@bf=\textbf\def\PYG@tc##1{\textcolor[rgb]{0.05,0.52,0.71}{##1}}}
\expandafter\def\csname PYG@tok@no\endcsname{\def\PYG@tc##1{\textcolor[rgb]{0.38,0.68,0.84}{##1}}}
\expandafter\def\csname PYG@tok@na\endcsname{\def\PYG@tc##1{\textcolor[rgb]{0.25,0.44,0.63}{##1}}}
\expandafter\def\csname PYG@tok@nb\endcsname{\def\PYG@tc##1{\textcolor[rgb]{0.00,0.44,0.13}{##1}}}
\expandafter\def\csname PYG@tok@nc\endcsname{\let\PYG@bf=\textbf\def\PYG@tc##1{\textcolor[rgb]{0.05,0.52,0.71}{##1}}}
\expandafter\def\csname PYG@tok@nd\endcsname{\let\PYG@bf=\textbf\def\PYG@tc##1{\textcolor[rgb]{0.33,0.33,0.33}{##1}}}
\expandafter\def\csname PYG@tok@ne\endcsname{\def\PYG@tc##1{\textcolor[rgb]{0.00,0.44,0.13}{##1}}}
\expandafter\def\csname PYG@tok@nf\endcsname{\def\PYG@tc##1{\textcolor[rgb]{0.02,0.16,0.49}{##1}}}
\expandafter\def\csname PYG@tok@si\endcsname{\let\PYG@it=\textit\def\PYG@tc##1{\textcolor[rgb]{0.44,0.63,0.82}{##1}}}
\expandafter\def\csname PYG@tok@s2\endcsname{\def\PYG@tc##1{\textcolor[rgb]{0.25,0.44,0.63}{##1}}}
\expandafter\def\csname PYG@tok@nt\endcsname{\let\PYG@bf=\textbf\def\PYG@tc##1{\textcolor[rgb]{0.02,0.16,0.45}{##1}}}
\expandafter\def\csname PYG@tok@nv\endcsname{\def\PYG@tc##1{\textcolor[rgb]{0.73,0.38,0.84}{##1}}}
\expandafter\def\csname PYG@tok@s1\endcsname{\def\PYG@tc##1{\textcolor[rgb]{0.25,0.44,0.63}{##1}}}
\expandafter\def\csname PYG@tok@ch\endcsname{\let\PYG@it=\textit\def\PYG@tc##1{\textcolor[rgb]{0.25,0.50,0.56}{##1}}}
\expandafter\def\csname PYG@tok@m\endcsname{\def\PYG@tc##1{\textcolor[rgb]{0.13,0.50,0.31}{##1}}}
\expandafter\def\csname PYG@tok@gp\endcsname{\let\PYG@bf=\textbf\def\PYG@tc##1{\textcolor[rgb]{0.78,0.36,0.04}{##1}}}
\expandafter\def\csname PYG@tok@sh\endcsname{\def\PYG@tc##1{\textcolor[rgb]{0.25,0.44,0.63}{##1}}}
\expandafter\def\csname PYG@tok@ow\endcsname{\let\PYG@bf=\textbf\def\PYG@tc##1{\textcolor[rgb]{0.00,0.44,0.13}{##1}}}
\expandafter\def\csname PYG@tok@sx\endcsname{\def\PYG@tc##1{\textcolor[rgb]{0.78,0.36,0.04}{##1}}}
\expandafter\def\csname PYG@tok@bp\endcsname{\def\PYG@tc##1{\textcolor[rgb]{0.00,0.44,0.13}{##1}}}
\expandafter\def\csname PYG@tok@c1\endcsname{\let\PYG@it=\textit\def\PYG@tc##1{\textcolor[rgb]{0.25,0.50,0.56}{##1}}}
\expandafter\def\csname PYG@tok@o\endcsname{\def\PYG@tc##1{\textcolor[rgb]{0.40,0.40,0.40}{##1}}}
\expandafter\def\csname PYG@tok@kc\endcsname{\let\PYG@bf=\textbf\def\PYG@tc##1{\textcolor[rgb]{0.00,0.44,0.13}{##1}}}
\expandafter\def\csname PYG@tok@c\endcsname{\let\PYG@it=\textit\def\PYG@tc##1{\textcolor[rgb]{0.25,0.50,0.56}{##1}}}
\expandafter\def\csname PYG@tok@mf\endcsname{\def\PYG@tc##1{\textcolor[rgb]{0.13,0.50,0.31}{##1}}}
\expandafter\def\csname PYG@tok@err\endcsname{\def\PYG@bc##1{\setlength{\fboxsep}{0pt}\fcolorbox[rgb]{1.00,0.00,0.00}{1,1,1}{\strut ##1}}}
\expandafter\def\csname PYG@tok@mb\endcsname{\def\PYG@tc##1{\textcolor[rgb]{0.13,0.50,0.31}{##1}}}
\expandafter\def\csname PYG@tok@ss\endcsname{\def\PYG@tc##1{\textcolor[rgb]{0.32,0.47,0.09}{##1}}}
\expandafter\def\csname PYG@tok@sr\endcsname{\def\PYG@tc##1{\textcolor[rgb]{0.14,0.33,0.53}{##1}}}
\expandafter\def\csname PYG@tok@mo\endcsname{\def\PYG@tc##1{\textcolor[rgb]{0.13,0.50,0.31}{##1}}}
\expandafter\def\csname PYG@tok@kd\endcsname{\let\PYG@bf=\textbf\def\PYG@tc##1{\textcolor[rgb]{0.00,0.44,0.13}{##1}}}
\expandafter\def\csname PYG@tok@mi\endcsname{\def\PYG@tc##1{\textcolor[rgb]{0.13,0.50,0.31}{##1}}}
\expandafter\def\csname PYG@tok@kn\endcsname{\let\PYG@bf=\textbf\def\PYG@tc##1{\textcolor[rgb]{0.00,0.44,0.13}{##1}}}
\expandafter\def\csname PYG@tok@cpf\endcsname{\let\PYG@it=\textit\def\PYG@tc##1{\textcolor[rgb]{0.25,0.50,0.56}{##1}}}
\expandafter\def\csname PYG@tok@kr\endcsname{\let\PYG@bf=\textbf\def\PYG@tc##1{\textcolor[rgb]{0.00,0.44,0.13}{##1}}}
\expandafter\def\csname PYG@tok@s\endcsname{\def\PYG@tc##1{\textcolor[rgb]{0.25,0.44,0.63}{##1}}}
\expandafter\def\csname PYG@tok@kp\endcsname{\def\PYG@tc##1{\textcolor[rgb]{0.00,0.44,0.13}{##1}}}
\expandafter\def\csname PYG@tok@w\endcsname{\def\PYG@tc##1{\textcolor[rgb]{0.73,0.73,0.73}{##1}}}
\expandafter\def\csname PYG@tok@kt\endcsname{\def\PYG@tc##1{\textcolor[rgb]{0.56,0.13,0.00}{##1}}}
\expandafter\def\csname PYG@tok@sc\endcsname{\def\PYG@tc##1{\textcolor[rgb]{0.25,0.44,0.63}{##1}}}
\expandafter\def\csname PYG@tok@sb\endcsname{\def\PYG@tc##1{\textcolor[rgb]{0.25,0.44,0.63}{##1}}}
\expandafter\def\csname PYG@tok@k\endcsname{\let\PYG@bf=\textbf\def\PYG@tc##1{\textcolor[rgb]{0.00,0.44,0.13}{##1}}}
\expandafter\def\csname PYG@tok@se\endcsname{\let\PYG@bf=\textbf\def\PYG@tc##1{\textcolor[rgb]{0.25,0.44,0.63}{##1}}}
\expandafter\def\csname PYG@tok@sd\endcsname{\let\PYG@it=\textit\def\PYG@tc##1{\textcolor[rgb]{0.25,0.44,0.63}{##1}}}

\def\PYGZbs{\char`\\}
\def\PYGZus{\char`\_}
\def\PYGZob{\char`\{}
\def\PYGZcb{\char`\}}
\def\PYGZca{\char`\^}
\def\PYGZam{\char`\&}
\def\PYGZlt{\char`\<}
\def\PYGZgt{\char`\>}
\def\PYGZsh{\char`\#}
\def\PYGZpc{\char`\%}
\def\PYGZdl{\char`\$}
\def\PYGZhy{\char`\-}
\def\PYGZsq{\char`\'}
\def\PYGZdq{\char`\"}
\def\PYGZti{\char`\~}
% for compatibility with earlier versions
\def\PYGZat{@}
\def\PYGZlb{[}
\def\PYGZrb{]}
\makeatother

\renewcommand\PYGZsq{\textquotesingle}

\begin{document}

\maketitle
\tableofcontents
\phantomsection\label{index::doc}



\chapter{Tutorials}
\label{tutorial:tutorials}\label{tutorial:welcome-to-zbuilder-s-documentation}\label{tutorial::doc}
zBuilder saves the current state of the ziva setup and allows you to re-build
scene amoung other things.  For example searching and replacing and using that
to mirror setups.  The basic idea is to build a ziva setup by hand and use these
scripts to save it.


\section{Saving and Loading Setup}
\label{tutorial:saving-and-loading-setup}
To build we need to instantiate a {\hyperref[zBuilder.setup:zBuilder.setup.Ziva.ZivaSetup]{\emph{\code{zBuilder.setup.Ziva.ZivaSetup}}}} object and we do it like so:

\begin{Verbatim}[commandchars=\\\{\}]
\PYG{k+kn}{import} \PYG{n+nn}{zBuilder.setup.Ziva} \PYG{k+kn}{as} \PYG{n+nn}{zva}
\PYG{n}{z} \PYG{o}{=} \PYG{n}{zva}\PYG{o}{.}\PYG{n}{ZivaSetup}\PYG{p}{(}\PYG{p}{)}
\end{Verbatim}

once we have that we need to fill it with what is in maya scene like so:

\begin{Verbatim}[commandchars=\\\{\}]
\PYG{n}{z}\PYG{o}{.}\PYG{n}{retrieve\PYGZus{}from\PYGZus{}scene}\PYG{p}{(}\PYG{p}{)}
\end{Verbatim}

This command works on selection.  It gets the setup from any ziva node selected including tissue or bone geo.  If nothing is selected it grabs a solver in the scene.

Once we have that we can do a few things.  One thing is to save on disk:

\begin{Verbatim}[commandchars=\\\{\}]
\PYG{n}{z}\PYG{o}{.}\PYG{n}{write}\PYG{p}{(}\PYG{l+s+s1}{\PYGZsq{}}\PYG{l+s+s1}{C:}\PYG{l+s+se}{\PYGZbs{}\PYGZbs{}}\PYG{l+s+s1}{Temp}\PYG{l+s+se}{\PYGZbs{}\PYGZbs{}}\PYG{l+s+s1}{test.ziva}\PYG{l+s+s1}{\PYGZsq{}}\PYG{p}{)}
\end{Verbatim}

all at once:

\begin{Verbatim}[commandchars=\\\{\}]
\PYG{k+kn}{import} \PYG{n+nn}{zBuilder.setup.Ziva} \PYG{k+kn}{as} \PYG{n+nn}{zva}
\PYG{n}{z} \PYG{o}{=} \PYG{n}{zva}\PYG{o}{.}\PYG{n}{ZivaSetup}\PYG{p}{(}\PYG{p}{)}
\PYG{n}{z}\PYG{o}{.}\PYG{n}{retrieve\PYGZus{}from\PYGZus{}scene}\PYG{p}{(}\PYG{p}{)}
\PYG{n}{z}\PYG{o}{.}\PYG{n}{write}\PYG{p}{(}\PYG{l+s+s1}{\PYGZsq{}}\PYG{l+s+s1}{C:}\PYG{l+s+se}{\PYGZbs{}\PYGZbs{}}\PYG{l+s+s1}{Temp}\PYG{l+s+se}{\PYGZbs{}\PYGZbs{}}\PYG{l+s+s1}{test.ziva}\PYG{l+s+s1}{\PYGZsq{}}\PYG{p}{)}
\end{Verbatim}

to load it from disk we retrieve it from file:

\begin{Verbatim}[commandchars=\\\{\}]
\PYG{k+kn}{import} \PYG{n+nn}{zBuilder.setup.Ziva} \PYG{k+kn}{as} \PYG{n+nn}{zva}
\PYG{n}{z} \PYG{o}{=} \PYG{n}{zva}\PYG{o}{.}\PYG{n}{ZivaSetup}\PYG{p}{(}\PYG{p}{)}
\PYG{n}{z}\PYG{o}{.}\PYG{n}{retrieve\PYGZus{}from\PYGZus{}file}\PYG{p}{(}\PYG{l+s+s1}{\PYGZsq{}}\PYG{l+s+s1}{C:}\PYG{l+s+se}{\PYGZbs{}\PYGZbs{}}\PYG{l+s+s1}{Temp}\PYG{l+s+se}{\PYGZbs{}\PYGZbs{}}\PYG{l+s+s1}{test.ziva}\PYG{l+s+s1}{\PYGZsq{}}\PYG{p}{)}
\end{Verbatim}


\section{Building Setup}
\label{tutorial:building-setup}
Assuming we have the `z' object loaded with setup we want, to build we do a:

\begin{Verbatim}[commandchars=\\\{\}]
\PYG{n}{z}\PYG{o}{.}\PYG{n}{apply}\PYG{p}{(}\PYG{p}{)}
\end{Verbatim}

note that this works on a clean scene with only the geo or an existing setup.
with an exisitng setup if it finds the same node it updates properties and maps.


\section{Maps}
\label{tutorial:maps}
this stores the mesh information so if you updated geo with different topology it
will interpolate maps in world space.  By default, it checks if it needs to
and if it doesn't it ignores this step.  To force it or turn it off use this flag:

\begin{Verbatim}[commandchars=\\\{\}]
\PYG{n}{z}\PYG{o}{.}\PYG{n}{apply}\PYG{p}{(}\PYG{n}{interp\PYGZus{}maps}\PYG{o}{=}\PYG{l+s+s1}{\PYGZsq{}}\PYG{l+s+s1}{auto}\PYG{l+s+s1}{\PYGZsq{}}\PYG{p}{)}
\end{Verbatim}

`'True'' = always interp maps
`'False'' = Never interp maps
`'auto'' = checks if needed


\section{Search and replace}
\label{tutorial:search-and-replace}
to search and replace you can do a:

\begin{Verbatim}[commandchars=\\\{\}]
\PYG{n}{z}\PYG{o}{.}\PYG{n}{string\PYGZus{}replace}\PYG{p}{(}\PYG{l+s+s1}{\PYGZsq{}}\PYG{l+s+s1}{r\PYGZus{}bicep\PYGZus{}muscle22}\PYG{l+s+s1}{\PYGZsq{}}\PYG{p}{,}\PYG{l+s+s1}{\PYGZsq{}}\PYG{l+s+s1}{r\PYGZus{}bicep\PYGZus{}muscle}\PYG{l+s+s1}{\PYGZsq{}}\PYG{p}{)}
\end{Verbatim}

That will replace all instances of \emph{r\_bicep\_muscle22} with \emph{r\_bicep\_muscle}
you can feed it regular expressions so this:

\begin{Verbatim}[commandchars=\\\{\}]
\PYG{n}{z}\PYG{o}{.}\PYG{n}{string\PYGZus{}replace}\PYG{p}{(}\PYG{l+s+s1}{\PYGZsq{}}\PYG{l+s+s1}{\PYGZca{}r\PYGZus{}}\PYG{l+s+s1}{\PYGZsq{}}\PYG{p}{,}\PYG{l+s+s1}{\PYGZsq{}}\PYG{l+s+s1}{l\PYGZus{}}\PYG{l+s+s1}{\PYGZsq{}}\PYG{p}{)}
\end{Verbatim}

will replace \emph{r\_} with \emph{l\_} IF it is at begining of line.


\section{Mirroring Setup}
\label{tutorial:mirroring-setup}
Earlier I showed you about retrieve\_from\_scene.  For mirroring it is best to use:

\begin{Verbatim}[commandchars=\\\{\}]
\PYG{n}{z}\PYG{o}{.}\PYG{n}{retrieve\PYGZus{}from\PYGZus{}scene\PYGZus{}selection}\PYG{p}{(}\PYG{p}{)}
\end{Verbatim}

That method will use selection to fill the data.  Use case is to select your left muscles for example and mirror them.  So lets try it:

\begin{Verbatim}[commandchars=\\\{\}]
\PYG{c+c1}{\PYGZsh{} select left muscles in scene}
\PYG{n}{z} \PYG{o}{=} \PYG{n}{zva}\PYG{o}{.}\PYG{n}{ZivaSetup}\PYG{p}{(}\PYG{p}{)}
\PYG{n}{z}\PYG{o}{.}\PYG{n}{retrieve\PYGZus{}from\PYGZus{}scene\PYGZus{}selection}\PYG{p}{(}\PYG{p}{)}
\PYG{n}{z}\PYG{o}{.}\PYG{n}{string\PYGZus{}replace}\PYG{p}{(}\PYG{l+s+s1}{\PYGZsq{}}\PYG{l+s+s1}{\PYGZca{}l\PYGZus{}}\PYG{l+s+s1}{\PYGZsq{}}\PYG{p}{,}\PYG{l+s+s1}{\PYGZsq{}}\PYG{l+s+s1}{r\PYGZus{}}\PYG{l+s+s1}{\PYGZsq{}}\PYG{p}{)}
\PYG{n}{z}\PYG{o}{.}\PYG{n}{apply}\PYG{p}{(}\PYG{p}{)}
\end{Verbatim}

What that does is put the left muscles in object and does searches for \emph{l\_} at begining of name and replaces with \emph{r\_}.  Sometimes you need to do a couple seach and replaces
and you do that like so:

\begin{Verbatim}[commandchars=\\\{\}]
\PYG{n}{z} \PYG{o}{=} \PYG{n}{zva}\PYG{o}{.}\PYG{n}{ZivaSetup}\PYG{p}{(}\PYG{p}{)}
\PYG{n}{z}\PYG{o}{.}\PYG{n}{retrieve\PYGZus{}from\PYGZus{}scene\PYGZus{}selection}\PYG{p}{(}\PYG{p}{)}
\PYG{n}{z}\PYG{o}{.}\PYG{n}{string\PYGZus{}replace}\PYG{p}{(}\PYG{l+s+s1}{\PYGZsq{}}\PYG{l+s+s1}{\PYGZca{}l\PYGZus{}}\PYG{l+s+s1}{\PYGZsq{}}\PYG{p}{,}\PYG{l+s+s1}{\PYGZsq{}}\PYG{l+s+s1}{r\PYGZus{}}\PYG{l+s+s1}{\PYGZsq{}}\PYG{p}{)}
\PYG{n}{z}\PYG{o}{.}\PYG{n}{string\PYGZus{}replace}\PYG{p}{(}\PYG{l+s+s1}{\PYGZsq{}}\PYG{l+s+s1}{\PYGZus{}l\PYGZus{}}\PYG{l+s+s1}{\PYGZsq{}}\PYG{p}{,}\PYG{l+s+s1}{\PYGZsq{}}\PYG{l+s+s1}{\PYGZus{}r\PYGZus{}}\PYG{l+s+s1}{\PYGZsq{}}\PYG{p}{)}
\PYG{n}{z}\PYG{o}{.}\PYG{n}{apply}\PYG{p}{(}\PYG{p}{)}
\end{Verbatim}

that will again replace \emph{l\_} with \emph{r\_} if at beinging of line AND replace \emph{\_l\_}
with \emph{\_r\_} andwhere it finds it.

currently for mirroring to work the zNodes need to be named with some naming
convention that can be search and replacable so it can identify opposite side.


\chapter{zBuilder}
\label{modules:zbuilder}\label{modules::doc}

\section{zBuilder.nodeCollection}
\label{zBuilder:zbuilder-nodecollection}\label{zBuilder::doc}\label{zBuilder:module-zBuilder.nodeCollection}\index{zBuilder.nodeCollection (module)}\index{BaseNodeEncoder (class in zBuilder.nodeCollection)}

\begin{fulllineitems}
\phantomsection\label{zBuilder:zBuilder.nodeCollection.BaseNodeEncoder}\pysiglinewithargsret{\strong{class }\code{zBuilder.nodeCollection.}\bfcode{BaseNodeEncoder}}{\emph{skipkeys=False}, \emph{ensure\_ascii=True}, \emph{check\_circular=True}, \emph{allow\_nan=True}, \emph{sort\_keys=False}, \emph{indent=None}, \emph{separators=None}, \emph{encoding='utf-8'}, \emph{default=None}}{}
Bases: \code{json.encoder.JSONEncoder}
\index{default() (zBuilder.nodeCollection.BaseNodeEncoder method)}

\begin{fulllineitems}
\phantomsection\label{zBuilder:zBuilder.nodeCollection.BaseNodeEncoder.default}\pysiglinewithargsret{\bfcode{default}}{\emph{obj}}{}
\end{fulllineitems}


\end{fulllineitems}

\index{NodeCollection (class in zBuilder.nodeCollection)}

\begin{fulllineitems}
\phantomsection\label{zBuilder:zBuilder.nodeCollection.NodeCollection}\pysigline{\strong{class }\code{zBuilder.nodeCollection.}\bfcode{NodeCollection}}
Bases: \code{object}
\index{add\_data() (zBuilder.nodeCollection.NodeCollection method)}

\begin{fulllineitems}
\phantomsection\label{zBuilder:zBuilder.nodeCollection.NodeCollection.add_data}\pysiglinewithargsret{\bfcode{add\_data}}{\emph{key}, \emph{name}, \emph{data=None}}{}
appends a mesh to the mesh list
\begin{quote}\begin{description}
\item[{Parameters}] \leavevmode\begin{itemize}
\item {} 
\textbf{\texttt{key}} (\emph{str}) -- places data in this key in dict.

\item {} 
\textbf{\texttt{name}} (\emph{str}) -- name of data to place.

\end{itemize}

\end{description}\end{quote}

\end{fulllineitems}

\index{add\_node() (zBuilder.nodeCollection.NodeCollection method)}

\begin{fulllineitems}
\phantomsection\label{zBuilder:zBuilder.nodeCollection.NodeCollection.add_node}\pysiglinewithargsret{\bfcode{add\_node}}{\emph{node}}{}
appends a node to the node list
\begin{quote}\begin{description}
\item[{Parameters}] \leavevmode
\textbf{\texttt{node}} (\emph{obj}) -- the node obj to append to collection list.

\end{description}\end{quote}

\end{fulllineitems}

\index{apply() (zBuilder.nodeCollection.NodeCollection method)}

\begin{fulllineitems}
\phantomsection\label{zBuilder:zBuilder.nodeCollection.NodeCollection.apply}\pysiglinewithargsret{\bfcode{apply}}{}{}
must create a method to inherit this class

\end{fulllineitems}

\index{compare() (zBuilder.nodeCollection.NodeCollection method)}

\begin{fulllineitems}
\phantomsection\label{zBuilder:zBuilder.nodeCollection.NodeCollection.compare}\pysiglinewithargsret{\bfcode{compare}}{\emph{type\_filter=None}, \emph{node\_filter=None}}{}
print info on each node
\begin{quote}\begin{description}
\item[{Parameters}] \leavevmode\begin{itemize}
\item {} 
\textbf{\texttt{type\_filter}} (\emph{str}) -- filter by node type.  Defaults to \textbf{None}

\item {} 
\textbf{\texttt{node\_filter}} (\emph{str}) -- filter by node name. Defaults to \textbf{None}

\item {} 
\textbf{\texttt{print\_data}} (\emph{bool}) -- prints name of data stored.  Defaults to \textbf{False}

\end{itemize}

\end{description}\end{quote}

\end{fulllineitems}

\index{data (zBuilder.nodeCollection.NodeCollection attribute)}

\begin{fulllineitems}
\phantomsection\label{zBuilder:zBuilder.nodeCollection.NodeCollection.data}\pysigline{\bfcode{data}\strong{ = None}}
DATTTTA

\end{fulllineitems}

\index{from\_json\_data() (zBuilder.nodeCollection.NodeCollection method)}

\begin{fulllineitems}
\phantomsection\label{zBuilder:zBuilder.nodeCollection.NodeCollection.from_json_data}\pysiglinewithargsret{\bfcode{from\_json\_data}}{\emph{data}}{}
Gets data out of json serilization

\end{fulllineitems}

\index{get\_data\_by\_key() (zBuilder.nodeCollection.NodeCollection method)}

\begin{fulllineitems}
\phantomsection\label{zBuilder:zBuilder.nodeCollection.NodeCollection.get_data_by_key}\pysiglinewithargsret{\bfcode{get\_data\_by\_key}}{\emph{key}}{}
Gets all data for given `key'
\begin{quote}\begin{description}
\item[{Parameters}] \leavevmode
\textbf{\texttt{key}} (\emph{str}) -- the key to get data from

\item[{Returns}] \leavevmode
of data objs

\item[{Return type}] \leavevmode
list

\end{description}\end{quote}
\paragraph{Example}

get\_data\_by\_key(`mesh')

\end{fulllineitems}

\index{get\_data\_by\_key\_name() (zBuilder.nodeCollection.NodeCollection method)}

\begin{fulllineitems}
\phantomsection\label{zBuilder:zBuilder.nodeCollection.NodeCollection.get_data_by_key_name}\pysiglinewithargsret{\bfcode{get\_data\_by\_key\_name}}{\emph{key}, \emph{name}}{}
Gets data given `key'
\begin{quote}\begin{description}
\item[{Parameters}] \leavevmode\begin{itemize}
\item {} 
\textbf{\texttt{key}} (\emph{str}) -- the key to get data from.

\item {} 
\textbf{\texttt{name}} (\emph{str}) -- name of the data.

\end{itemize}

\item[{Returns}] \leavevmode
Object of data.

\item[{Return type}] \leavevmode
obj

\end{description}\end{quote}
\paragraph{Example}

\begin{Verbatim}[commandchars=\\\{\}]
\PYG{g+gp}{\PYGZgt{}\PYGZgt{}\PYGZgt{} }\PYG{n}{get\PYGZus{}data\PYGZus{}by\PYGZus{}key\PYGZus{}name}\PYG{p}{(}\PYG{l+s+s1}{\PYGZsq{}}\PYG{l+s+s1}{mesh}\PYG{l+s+s1}{\PYGZsq{}}\PYG{p}{,}\PYG{l+s+s1}{\PYGZsq{}}\PYG{l+s+s1}{l\PYGZus{}bicepMuscle}\PYG{l+s+s1}{\PYGZsq{}}\PYG{p}{)}
\end{Verbatim}

\end{fulllineitems}

\index{get\_json\_data() (zBuilder.nodeCollection.NodeCollection method)}

\begin{fulllineitems}
\phantomsection\label{zBuilder:zBuilder.nodeCollection.NodeCollection.get_json_data}\pysiglinewithargsret{\bfcode{get\_json\_data}}{}{}
Utility function to define data stored in json

\end{fulllineitems}

\index{get\_nodes() (zBuilder.nodeCollection.NodeCollection method)}

\begin{fulllineitems}
\phantomsection\label{zBuilder:zBuilder.nodeCollection.NodeCollection.get_nodes}\pysiglinewithargsret{\bfcode{get\_nodes}}{\emph{type\_filter=None}, \emph{node\_filter=None}}{}
get nodes in data object
\begin{quote}\begin{description}
\item[{Parameters}] \leavevmode\begin{itemize}
\item {} 
\textbf{\texttt{type\_filter}} (\emph{str}) -- filter by node type.  Defaults to \textbf{None}

\item {} 
\textbf{\texttt{node\_filter}} (\emph{str}) -- filter by node name.  Defaults to \textbf{None}

\end{itemize}

\item[{Returns}] \leavevmode
{[}{]} of nodes

\end{description}\end{quote}

\end{fulllineitems}

\index{print\_() (zBuilder.nodeCollection.NodeCollection method)}

\begin{fulllineitems}
\phantomsection\label{zBuilder:zBuilder.nodeCollection.NodeCollection.print_}\pysiglinewithargsret{\bfcode{print\_}}{\emph{type\_filter=None}, \emph{node\_filter=None}, \emph{print\_data=False}}{}
print info on each node
\begin{quote}\begin{description}
\item[{Parameters}] \leavevmode\begin{itemize}
\item {} 
\textbf{\texttt{type\_filter}} (\emph{str}) -- filter by node type.  Defaults to None

\item {} 
\textbf{\texttt{node\_filter}} (\emph{str}) -- filter by node name. Defaults to None

\item {} 
\textbf{\texttt{print\_data}} (\emph{bool}) -- prints name of data stored.  Defaults to False

\end{itemize}

\end{description}\end{quote}

\end{fulllineitems}

\index{retrieve\_from\_file() (zBuilder.nodeCollection.NodeCollection method)}

\begin{fulllineitems}
\phantomsection\label{zBuilder:zBuilder.nodeCollection.NodeCollection.retrieve_from_file}\pysiglinewithargsret{\bfcode{retrieve\_from\_file}}{\emph{filepath}}{}
reads data from a file
\begin{quote}\begin{description}
\item[{Parameters}] \leavevmode
\textbf{\texttt{filepath}} (\emph{str}) -- filepath to read from disk

\item[{Raises}] \leavevmode
\code{IOError} --
If not able to read file

\end{description}\end{quote}

\end{fulllineitems}

\index{retrieve\_from\_scene() (zBuilder.nodeCollection.NodeCollection method)}

\begin{fulllineitems}
\phantomsection\label{zBuilder:zBuilder.nodeCollection.NodeCollection.retrieve_from_scene}\pysiglinewithargsret{\bfcode{retrieve\_from\_scene}}{\emph{selection}}{}
must create a method to inherit this class

\end{fulllineitems}

\index{stats() (zBuilder.nodeCollection.NodeCollection method)}

\begin{fulllineitems}
\phantomsection\label{zBuilder:zBuilder.nodeCollection.NodeCollection.stats}\pysiglinewithargsret{\bfcode{stats}}{\emph{type\_filter=None}}{}
prints out basic stats on data
\begin{quote}\begin{description}
\item[{Parameters}] \leavevmode
\textbf{\texttt{type\_filter}} (\emph{str}) -- filter by node type.  Defaults to None

\end{description}\end{quote}

\end{fulllineitems}

\index{string\_replace() (zBuilder.nodeCollection.NodeCollection method)}

\begin{fulllineitems}
\phantomsection\label{zBuilder:zBuilder.nodeCollection.NodeCollection.string_replace}\pysiglinewithargsret{\bfcode{string\_replace}}{\emph{search}, \emph{replace}}{}
searches and replaces with regular expressions items in data
\begin{quote}\begin{description}
\item[{Parameters}] \leavevmode\begin{itemize}
\item {} 
\textbf{\texttt{search}} (\emph{str}) -- what to search for

\item {} 
\textbf{\texttt{replace}} (\emph{str}) -- what to replace it with

\end{itemize}

\end{description}\end{quote}
\paragraph{Example}

replace \emph{r\_} at front of item with \emph{l\_}:

\begin{Verbatim}[commandchars=\\\{\}]
\PYG{g+gp}{\PYGZgt{}\PYGZgt{}\PYGZgt{} }\PYG{n}{z}\PYG{o}{.}\PYG{n}{string\PYGZus{}replace}\PYG{p}{(}\PYG{l+s+s1}{\PYGZsq{}}\PYG{l+s+s1}{\PYGZca{}r\PYGZus{}}\PYG{l+s+s1}{\PYGZsq{}}\PYG{p}{,}\PYG{l+s+s1}{\PYGZsq{}}\PYG{l+s+s1}{l\PYGZus{}}\PYG{l+s+s1}{\PYGZsq{}}\PYG{p}{)}
\end{Verbatim}

replace \emph{\_r} at end of line with \emph{\_l}:

\begin{Verbatim}[commandchars=\\\{\}]
\PYG{g+gp}{\PYGZgt{}\PYGZgt{}\PYGZgt{} }\PYG{n}{z}\PYG{o}{.}\PYG{n}{string\PYGZus{}replace}\PYG{p}{(}\PYG{l+s+s1}{\PYGZsq{}}\PYG{l+s+s1}{\PYGZus{}r\PYGZdl{}}\PYG{l+s+s1}{\PYGZsq{}}\PYG{p}{,}\PYG{l+s+s1}{\PYGZsq{}}\PYG{l+s+s1}{\PYGZus{}l}\PYG{l+s+s1}{\PYGZsq{}}\PYG{p}{)}
\end{Verbatim}

\end{fulllineitems}

\index{write() (zBuilder.nodeCollection.NodeCollection method)}

\begin{fulllineitems}
\phantomsection\label{zBuilder:zBuilder.nodeCollection.NodeCollection.write}\pysiglinewithargsret{\bfcode{write}}{\emph{filepath}}{}
writes data to disk in json format
\begin{quote}\begin{description}
\item[{Parameters}] \leavevmode
\textbf{\texttt{filepath}} (\emph{str}) -- filepath to write to disk

\item[{Raises}] \leavevmode
\code{IOError} --
If not able to write file

\end{description}\end{quote}

\end{fulllineitems}


\end{fulllineitems}

\index{load\_base\_node() (in module zBuilder.nodeCollection)}

\begin{fulllineitems}
\phantomsection\label{zBuilder:zBuilder.nodeCollection.load_base_node}\pysiglinewithargsret{\code{zBuilder.nodeCollection.}\bfcode{load\_base\_node}}{\emph{json\_object}}{}
Loads json objects into proper classes
\begin{quote}\begin{description}
\item[{Parameters}] \leavevmode
\textbf{\texttt{json\_object}} (\emph{obj}) -- json obj to perform action on

\item[{Returns}] \leavevmode
Result of operation

\item[{Return type}] \leavevmode
obj

\end{description}\end{quote}

\end{fulllineitems}

\index{replace\_dict\_keys() (in module zBuilder.nodeCollection)}

\begin{fulllineitems}
\phantomsection\label{zBuilder:zBuilder.nodeCollection.replace_dict_keys}\pysiglinewithargsret{\code{zBuilder.nodeCollection.}\bfcode{replace\_dict\_keys}}{\emph{search}, \emph{replace}, \emph{dictionary}}{}
Does a search and replace on dictionary keys
\begin{quote}\begin{description}
\item[{Parameters}] \leavevmode\begin{itemize}
\item {} 
\textbf{\texttt{search}} (\emph{str}) -- search term

\item {} 
\textbf{\texttt{replace}} (\emph{str}) -- replace term

\item {} 
\textbf{\texttt{dictionary}} (\emph{dict}) -- the dictionary to do search on

\end{itemize}

\item[{Returns}] \leavevmode
result of search and replace

\item[{Return type}] \leavevmode
dict

\end{description}\end{quote}

\end{fulllineitems}

\index{replace\_longname() (in module zBuilder.nodeCollection)}

\begin{fulllineitems}
\phantomsection\label{zBuilder:zBuilder.nodeCollection.replace_longname}\pysiglinewithargsret{\code{zBuilder.nodeCollection.}\bfcode{replace\_longname}}{\emph{search}, \emph{replace}, \emph{longName}}{}
does a search and replace on a long name.  It splits it up by (`\textbar{}') then
performs it on each piece
\begin{quote}\begin{description}
\item[{Parameters}] \leavevmode\begin{itemize}
\item {} 
\textbf{\texttt{search}} (\emph{str}) -- search term

\item {} 
\textbf{\texttt{replace}} (\emph{str}) -- replace term

\item {} 
\textbf{\texttt{longName}} (\emph{str}) -- the long name to perform action on

\end{itemize}

\item[{Returns}] \leavevmode
result of search and replace

\item[{Return type}] \leavevmode
str

\end{description}\end{quote}

\end{fulllineitems}

\index{str\_to\_class() (in module zBuilder.nodeCollection)}

\begin{fulllineitems}
\phantomsection\label{zBuilder:zBuilder.nodeCollection.str_to_class}\pysiglinewithargsret{\code{zBuilder.nodeCollection.}\bfcode{str\_to\_class}}{\emph{module}, \emph{name}}{}
Given module and name instantiantes a class
\begin{quote}\begin{description}
\item[{Parameters}] \leavevmode\begin{itemize}
\item {} 
\textbf{\texttt{module}} (\emph{str}) -- module

\item {} 
\textbf{\texttt{name}} (\emph{str}) -- the class name

\end{itemize}

\item[{Returns}] \leavevmode
class object

\item[{Return type}] \leavevmode
obj

\end{description}\end{quote}

\end{fulllineitems}

\index{time\_this() (in module zBuilder.nodeCollection)}

\begin{fulllineitems}
\phantomsection\label{zBuilder:zBuilder.nodeCollection.time_this}\pysiglinewithargsret{\code{zBuilder.nodeCollection.}\bfcode{time\_this}}{\emph{original\_function}}{}
\end{fulllineitems}



\subsection{zBuilder.setup}
\label{zBuilder.setup::doc}\label{zBuilder.setup:zbuilder-setup}

\subsubsection{zBuilder.setup.Attributes}
\label{zBuilder.setup:module-zBuilder.setup.Attributes}\label{zBuilder.setup:zbuilder-setup-attributes}\index{zBuilder.setup.Attributes (module)}\index{AttributesSetup (class in zBuilder.setup.Attributes)}

\begin{fulllineitems}
\phantomsection\label{zBuilder.setup:zBuilder.setup.Attributes.AttributesSetup}\pysigline{\strong{class }\code{zBuilder.setup.Attributes.}\bfcode{AttributesSetup}}
Bases: {\hyperref[zBuilder:zBuilder.nodeCollection.NodeCollection]{\emph{\code{zBuilder.nodeCollection.NodeCollection}}}}
\index{apply() (zBuilder.setup.Attributes.AttributesSetup method)}

\begin{fulllineitems}
\phantomsection\label{zBuilder.setup:zBuilder.setup.Attributes.AttributesSetup.apply}\pysiglinewithargsret{\bfcode{apply}}{\emph{*args}, \emph{**kwargs}}{}
\end{fulllineitems}

\index{retrieve\_from\_scene() (zBuilder.setup.Attributes.AttributesSetup method)}

\begin{fulllineitems}
\phantomsection\label{zBuilder.setup:zBuilder.setup.Attributes.AttributesSetup.retrieve_from_scene}\pysiglinewithargsret{\bfcode{retrieve\_from\_scene}}{\emph{*args}, \emph{**kwargs}}{}
\end{fulllineitems}


\end{fulllineitems}



\subsubsection{zBuilder.setup.Constraints}
\label{zBuilder.setup:zbuilder-setup-constraints}\label{zBuilder.setup:module-zBuilder.setup.Constraints}\index{zBuilder.setup.Constraints (module)}\index{ConstraintsSetup (class in zBuilder.setup.Constraints)}

\begin{fulllineitems}
\phantomsection\label{zBuilder.setup:zBuilder.setup.Constraints.ConstraintsSetup}\pysigline{\strong{class }\code{zBuilder.setup.Constraints.}\bfcode{ConstraintsSetup}}
Bases: {\hyperref[zBuilder:zBuilder.nodeCollection.NodeCollection]{\emph{\code{zBuilder.nodeCollection.NodeCollection}}}}
\index{apply() (zBuilder.setup.Constraints.ConstraintsSetup method)}

\begin{fulllineitems}
\phantomsection\label{zBuilder.setup:zBuilder.setup.Constraints.ConstraintsSetup.apply}\pysiglinewithargsret{\bfcode{apply}}{}{}
\end{fulllineitems}

\index{get\_source() (zBuilder.setup.Constraints.ConstraintsSetup method)}

\begin{fulllineitems}
\phantomsection\label{zBuilder.setup:zBuilder.setup.Constraints.ConstraintsSetup.get_source}\pysiglinewithargsret{\bfcode{get\_source}}{\emph{node}}{}
\end{fulllineitems}

\index{get\_target() (zBuilder.setup.Constraints.ConstraintsSetup method)}

\begin{fulllineitems}
\phantomsection\label{zBuilder.setup:zBuilder.setup.Constraints.ConstraintsSetup.get_target}\pysiglinewithargsret{\bfcode{get\_target}}{\emph{node}}{}
\end{fulllineitems}

\index{retrieve\_from\_scene() (zBuilder.setup.Constraints.ConstraintsSetup method)}

\begin{fulllineitems}
\phantomsection\label{zBuilder.setup:zBuilder.setup.Constraints.ConstraintsSetup.retrieve_from_scene}\pysiglinewithargsret{\bfcode{retrieve\_from\_scene}}{}{}
\end{fulllineitems}


\end{fulllineitems}



\subsubsection{zBuilder.setup.Selection}
\label{zBuilder.setup:module-zBuilder.setup.Selection}\label{zBuilder.setup:zbuilder-setup-selection}\index{zBuilder.setup.Selection (module)}\index{SelectionSetup (class in zBuilder.setup.Selection)}

\begin{fulllineitems}
\phantomsection\label{zBuilder.setup:zBuilder.setup.Selection.SelectionSetup}\pysigline{\strong{class }\code{zBuilder.setup.Selection.}\bfcode{SelectionSetup}}
Bases: {\hyperref[zBuilder:zBuilder.nodeCollection.NodeCollection]{\emph{\code{zBuilder.nodeCollection.NodeCollection}}}}
\index{apply() (zBuilder.setup.Selection.SelectionSetup method)}

\begin{fulllineitems}
\phantomsection\label{zBuilder.setup:zBuilder.setup.Selection.SelectionSetup.apply}\pysiglinewithargsret{\bfcode{apply}}{}{}
\end{fulllineitems}

\index{retrieve\_from\_scene() (zBuilder.setup.Selection.SelectionSetup method)}

\begin{fulllineitems}
\phantomsection\label{zBuilder.setup:zBuilder.setup.Selection.SelectionSetup.retrieve_from_scene}\pysiglinewithargsret{\bfcode{retrieve\_from\_scene}}{\emph{selection}}{}
\end{fulllineitems}

\index{return\_selection() (zBuilder.setup.Selection.SelectionSetup method)}

\begin{fulllineitems}
\phantomsection\label{zBuilder.setup:zBuilder.setup.Selection.SelectionSetup.return_selection}\pysiglinewithargsret{\bfcode{return\_selection}}{}{}
\end{fulllineitems}


\end{fulllineitems}



\subsubsection{zBuilder.setup.Ziva}
\label{zBuilder.setup:module-zBuilder.setup.Ziva}\label{zBuilder.setup:zbuilder-setup-ziva}\index{zBuilder.setup.Ziva (module)}\index{ZivaSetup (class in zBuilder.setup.Ziva)}

\begin{fulllineitems}
\phantomsection\label{zBuilder.setup:zBuilder.setup.Ziva.ZivaSetup}\pysigline{\strong{class }\code{zBuilder.setup.Ziva.}\bfcode{ZivaSetup}}
Bases: {\hyperref[zBuilder:zBuilder.nodeCollection.NodeCollection]{\emph{\code{zBuilder.nodeCollection.NodeCollection}}}}

To capture a ziva setup
\index{apply() (zBuilder.setup.Ziva.ZivaSetup method)}

\begin{fulllineitems}
\phantomsection\label{zBuilder.setup:zBuilder.setup.Ziva.ZivaSetup.apply}\pysiglinewithargsret{\bfcode{apply}}{\emph{*args}, \emph{**kwargs}}{}
\end{fulllineitems}

\index{mirror() (zBuilder.setup.Ziva.ZivaSetup method)}

\begin{fulllineitems}
\phantomsection\label{zBuilder.setup:zBuilder.setup.Ziva.ZivaSetup.mirror}\pysiglinewithargsret{\bfcode{mirror}}{\emph{search}, \emph{replace}}{}
\end{fulllineitems}

\index{retrieve\_from\_scene() (zBuilder.setup.Ziva.ZivaSetup method)}

\begin{fulllineitems}
\phantomsection\label{zBuilder.setup:zBuilder.setup.Ziva.ZivaSetup.retrieve_from_scene}\pysiglinewithargsret{\bfcode{retrieve\_from\_scene}}{\emph{*args}, \emph{**kwargs}}{}
\end{fulllineitems}

\index{retrieve\_from\_scene\_selection() (zBuilder.setup.Ziva.ZivaSetup method)}

\begin{fulllineitems}
\phantomsection\label{zBuilder.setup:zBuilder.setup.Ziva.ZivaSetup.retrieve_from_scene_selection}\pysiglinewithargsret{\bfcode{retrieve\_from\_scene\_selection}}{\emph{*args}, \emph{**kwargs}}{}
\end{fulllineitems}


\end{fulllineitems}



\subsection{zBuilder.data package}
\label{zBuilder.data::doc}\label{zBuilder.data:zbuilder-data-package}

\subsubsection{Submodules}
\label{zBuilder.data:submodules}

\subsubsection{zBuilder.data.map module}
\label{zBuilder.data:zbuilder-data-map-module}\label{zBuilder.data:module-zBuilder.data.map}\index{zBuilder.data.map (module)}\index{Map (class in zBuilder.data.map)}

\begin{fulllineitems}
\phantomsection\label{zBuilder.data:zBuilder.data.map.Map}\pysigline{\strong{class }\code{zBuilder.data.map.}\bfcode{Map}}
Bases: \code{object}
\index{get\_mesh() (zBuilder.data.map.Map method)}

\begin{fulllineitems}
\phantomsection\label{zBuilder.data:zBuilder.data.map.Map.get_mesh}\pysiglinewithargsret{\bfcode{get\_mesh}}{\emph{longName=False}}{}
\end{fulllineitems}

\index{get\_name() (zBuilder.data.map.Map method)}

\begin{fulllineitems}
\phantomsection\label{zBuilder.data:zBuilder.data.map.Map.get_name}\pysiglinewithargsret{\bfcode{get\_name}}{\emph{longName=False}}{}
\end{fulllineitems}

\index{get\_value() (zBuilder.data.map.Map method)}

\begin{fulllineitems}
\phantomsection\label{zBuilder.data:zBuilder.data.map.Map.get_value}\pysiglinewithargsret{\bfcode{get\_value}}{}{}
\end{fulllineitems}

\index{set\_mesh() (zBuilder.data.map.Map method)}

\begin{fulllineitems}
\phantomsection\label{zBuilder.data:zBuilder.data.map.Map.set_mesh}\pysiglinewithargsret{\bfcode{set\_mesh}}{\emph{mesh}}{}
\end{fulllineitems}

\index{set\_name() (zBuilder.data.map.Map method)}

\begin{fulllineitems}
\phantomsection\label{zBuilder.data:zBuilder.data.map.Map.set_name}\pysiglinewithargsret{\bfcode{set\_name}}{\emph{name}}{}
\end{fulllineitems}

\index{set\_value() (zBuilder.data.map.Map method)}

\begin{fulllineitems}
\phantomsection\label{zBuilder.data:zBuilder.data.map.Map.set_value}\pysiglinewithargsret{\bfcode{set\_value}}{\emph{value}}{}
\end{fulllineitems}

\index{string\_replace() (zBuilder.data.map.Map method)}

\begin{fulllineitems}
\phantomsection\label{zBuilder.data:zBuilder.data.map.Map.string_replace}\pysiglinewithargsret{\bfcode{string\_replace}}{\emph{search}, \emph{replace}}{}
\end{fulllineitems}


\end{fulllineitems}

\index{getMDagPathFromMeshName() (in module zBuilder.data.map)}

\begin{fulllineitems}
\phantomsection\label{zBuilder.data:zBuilder.data.map.getMDagPathFromMeshName}\pysiglinewithargsret{\code{zBuilder.data.map.}\bfcode{getMDagPathFromMeshName}}{\emph{meshName}}{}
\end{fulllineitems}

\index{get\_map\_data() (in module zBuilder.data.map)}

\begin{fulllineitems}
\phantomsection\label{zBuilder.data:zBuilder.data.map.get_map_data}\pysiglinewithargsret{\code{zBuilder.data.map.}\bfcode{get\_map\_data}}{\emph{node}, \emph{attr}, \emph{mesh}}{}
\end{fulllineitems}

\index{get\_mesh\_connectivity() (in module zBuilder.data.map)}

\begin{fulllineitems}
\phantomsection\label{zBuilder.data:zBuilder.data.map.get_mesh_connectivity}\pysiglinewithargsret{\code{zBuilder.data.map.}\bfcode{get\_mesh\_connectivity}}{\emph{mesh\_name}}{}
\end{fulllineitems}

\index{get\_weights() (in module zBuilder.data.map)}

\begin{fulllineitems}
\phantomsection\label{zBuilder.data:zBuilder.data.map.get_weights}\pysiglinewithargsret{\code{zBuilder.data.map.}\bfcode{get\_weights}}{\emph{node}, \emph{mesh}, \emph{attr}}{}
\end{fulllineitems}

\index{interpolateValues() (in module zBuilder.data.map)}

\begin{fulllineitems}
\phantomsection\label{zBuilder.data:zBuilder.data.map.interpolateValues}\pysiglinewithargsret{\code{zBuilder.data.map.}\bfcode{interpolateValues}}{\emph{sourceMeshName}, \emph{destinationMeshName}, \emph{wList}}{}~\begin{description}
\item[{Description:}] \leavevmode
Will transfer values between similar meshes with differing topology.
Lerps values from triangleIndex of closest point on mesh.

\item[{Accepts:}] \leavevmode
sourceMeshName, destinationMeshName - strings for each mesh transform

\end{description}

Returns:

\end{fulllineitems}

\index{replace\_longname() (in module zBuilder.data.map)}

\begin{fulllineitems}
\phantomsection\label{zBuilder.data:zBuilder.data.map.replace_longname}\pysiglinewithargsret{\code{zBuilder.data.map.}\bfcode{replace\_longname}}{\emph{search}, \emph{replace}, \emph{longName}}{}
does a search and replace on a long name.  It splits it up by (`\textbar{}') then
performs it on each piece
\begin{quote}\begin{description}
\item[{Parameters}] \leavevmode\begin{itemize}
\item {} 
\textbf{\texttt{search}} (\emph{str}) -- search term

\item {} 
\textbf{\texttt{replace}} (\emph{str}) -- replace term

\item {} 
\textbf{\texttt{longName}} (\emph{str}) -- the long name to perform action on

\end{itemize}

\item[{Returns}] \leavevmode
result of search and replace

\item[{Return type}] \leavevmode
str

\end{description}\end{quote}

\end{fulllineitems}

\index{set\_weights() (in module zBuilder.data.map)}

\begin{fulllineitems}
\phantomsection\label{zBuilder.data:zBuilder.data.map.set_weights}\pysiglinewithargsret{\code{zBuilder.data.map.}\bfcode{set\_weights}}{\emph{nodes}, \emph{data}, \emph{interp\_maps=False}}{}
\end{fulllineitems}



\subsubsection{zBuilder.data.mesh module}
\label{zBuilder.data:module-zBuilder.data.mesh}\label{zBuilder.data:zbuilder-data-mesh-module}\index{zBuilder.data.mesh (module)}\index{Mesh (class in zBuilder.data.mesh)}

\begin{fulllineitems}
\phantomsection\label{zBuilder.data:zBuilder.data.mesh.Mesh}\pysigline{\strong{class }\code{zBuilder.data.mesh.}\bfcode{Mesh}}
Bases: \code{object}
\index{build() (zBuilder.data.mesh.Mesh method)}

\begin{fulllineitems}
\phantomsection\label{zBuilder.data:zBuilder.data.mesh.Mesh.build}\pysiglinewithargsret{\bfcode{build}}{}{}
\end{fulllineitems}

\index{get\_name() (zBuilder.data.mesh.Mesh method)}

\begin{fulllineitems}
\phantomsection\label{zBuilder.data:zBuilder.data.mesh.Mesh.get_name}\pysiglinewithargsret{\bfcode{get\_name}}{\emph{longName=False}}{}
\end{fulllineitems}

\index{get\_point\_list() (zBuilder.data.mesh.Mesh method)}

\begin{fulllineitems}
\phantomsection\label{zBuilder.data:zBuilder.data.mesh.Mesh.get_point_list}\pysiglinewithargsret{\bfcode{get\_point\_list}}{}{}
\end{fulllineitems}

\index{get\_polygon\_connects() (zBuilder.data.mesh.Mesh method)}

\begin{fulllineitems}
\phantomsection\label{zBuilder.data:zBuilder.data.mesh.Mesh.get_polygon_connects}\pysiglinewithargsret{\bfcode{get\_polygon\_connects}}{}{}
\end{fulllineitems}

\index{get\_polygon\_counts() (zBuilder.data.mesh.Mesh method)}

\begin{fulllineitems}
\phantomsection\label{zBuilder.data:zBuilder.data.mesh.Mesh.get_polygon_counts}\pysiglinewithargsret{\bfcode{get\_polygon\_counts}}{}{}
\end{fulllineitems}

\index{set\_name() (zBuilder.data.mesh.Mesh method)}

\begin{fulllineitems}
\phantomsection\label{zBuilder.data:zBuilder.data.mesh.Mesh.set_name}\pysiglinewithargsret{\bfcode{set\_name}}{\emph{name}}{}
\end{fulllineitems}

\index{set\_point\_list() (zBuilder.data.mesh.Mesh method)}

\begin{fulllineitems}
\phantomsection\label{zBuilder.data:zBuilder.data.mesh.Mesh.set_point_list}\pysiglinewithargsret{\bfcode{set\_point\_list}}{\emph{pointList}}{}
\end{fulllineitems}

\index{set\_polygon\_connects() (zBuilder.data.mesh.Mesh method)}

\begin{fulllineitems}
\phantomsection\label{zBuilder.data:zBuilder.data.mesh.Mesh.set_polygon_connects}\pysiglinewithargsret{\bfcode{set\_polygon\_connects}}{\emph{pConnectList}}{}
\end{fulllineitems}

\index{set\_polygon\_counts() (zBuilder.data.mesh.Mesh method)}

\begin{fulllineitems}
\phantomsection\label{zBuilder.data:zBuilder.data.mesh.Mesh.set_polygon_counts}\pysiglinewithargsret{\bfcode{set\_polygon\_counts}}{\emph{pCountList}}{}
\end{fulllineitems}

\index{string\_replace() (zBuilder.data.mesh.Mesh method)}

\begin{fulllineitems}
\phantomsection\label{zBuilder.data:zBuilder.data.mesh.Mesh.string_replace}\pysiglinewithargsret{\bfcode{string\_replace}}{\emph{search}, \emph{replace}}{}
\end{fulllineitems}


\end{fulllineitems}

\index{buildMesh() (in module zBuilder.data.mesh)}

\begin{fulllineitems}
\phantomsection\label{zBuilder.data:zBuilder.data.mesh.buildMesh}\pysiglinewithargsret{\code{zBuilder.data.mesh.}\bfcode{buildMesh}}{\emph{name}, \emph{polygonCounts}, \emph{polygonConnects}, \emph{vertexArray}}{}
\end{fulllineitems}

\index{getMDagPathFromMeshName() (in module zBuilder.data.mesh)}

\begin{fulllineitems}
\phantomsection\label{zBuilder.data:zBuilder.data.mesh.getMDagPathFromMeshName}\pysiglinewithargsret{\code{zBuilder.data.mesh.}\bfcode{getMDagPathFromMeshName}}{\emph{meshName}}{}
\end{fulllineitems}

\index{get\_mesh\_connectivity() (in module zBuilder.data.mesh)}

\begin{fulllineitems}
\phantomsection\label{zBuilder.data:zBuilder.data.mesh.get_mesh_connectivity}\pysiglinewithargsret{\code{zBuilder.data.mesh.}\bfcode{get\_mesh\_connectivity}}{\emph{mesh\_name}}{}
\end{fulllineitems}

\index{get\_mesh\_data() (in module zBuilder.data.mesh)}

\begin{fulllineitems}
\phantomsection\label{zBuilder.data:zBuilder.data.mesh.get_mesh_data}\pysiglinewithargsret{\code{zBuilder.data.mesh.}\bfcode{get\_mesh\_data}}{\emph{meshName}}{}
\end{fulllineitems}

\index{replace\_longname() (in module zBuilder.data.mesh)}

\begin{fulllineitems}
\phantomsection\label{zBuilder.data:zBuilder.data.mesh.replace_longname}\pysiglinewithargsret{\code{zBuilder.data.mesh.}\bfcode{replace\_longname}}{\emph{search}, \emph{replace}, \emph{longName}}{}
does a search and replace on a long name.  It splits it up by (`\textbar{}') then
performs it on each piece
\begin{quote}\begin{description}
\item[{Parameters}] \leavevmode\begin{itemize}
\item {} 
\textbf{\texttt{search}} (\emph{str}) -- search term

\item {} 
\textbf{\texttt{replace}} (\emph{str}) -- replace term

\item {} 
\textbf{\texttt{longName}} (\emph{str}) -- the long name to perform action on

\end{itemize}

\item[{Returns}] \leavevmode
result of search and replace

\item[{Return type}] \leavevmode
str

\end{description}\end{quote}

\end{fulllineitems}



\subsubsection{Module contents}
\label{zBuilder.data:module-contents}\label{zBuilder.data:module-zBuilder.data}\index{zBuilder.data (module)}

\subsection{zBuilder.nodes}
\label{zBuilder.nodes::doc}\label{zBuilder.nodes:zbuilder-nodes}

\subsubsection{zBuilder.nodes.base}
\label{zBuilder.nodes:zbuilder-nodes-base}\label{zBuilder.nodes:module-zBuilder.nodes.base}\index{zBuilder.nodes.base (module)}\index{BaseNode (class in zBuilder.nodes.base)}

\begin{fulllineitems}
\phantomsection\label{zBuilder.nodes:zBuilder.nodes.base.BaseNode}\pysigline{\strong{class }\code{zBuilder.nodes.base.}\bfcode{BaseNode}}
Bases: \code{object}
\index{compare() (zBuilder.nodes.base.BaseNode method)}

\begin{fulllineitems}
\phantomsection\label{zBuilder.nodes:zBuilder.nodes.base.BaseNode.compare}\pysiglinewithargsret{\bfcode{compare}}{}{}
\end{fulllineitems}

\index{get\_association() (zBuilder.nodes.base.BaseNode method)}

\begin{fulllineitems}
\phantomsection\label{zBuilder.nodes:zBuilder.nodes.base.BaseNode.get_association}\pysiglinewithargsret{\bfcode{get\_association}}{\emph{longName=False}}{}
\end{fulllineitems}

\index{get\_attr\_key() (zBuilder.nodes.base.BaseNode method)}

\begin{fulllineitems}
\phantomsection\label{zBuilder.nodes:zBuilder.nodes.base.BaseNode.get_attr_key}\pysiglinewithargsret{\bfcode{get\_attr\_key}}{\emph{key}}{}
\end{fulllineitems}

\index{get\_attr\_key\_value() (zBuilder.nodes.base.BaseNode method)}

\begin{fulllineitems}
\phantomsection\label{zBuilder.nodes:zBuilder.nodes.base.BaseNode.get_attr_key_value}\pysiglinewithargsret{\bfcode{get\_attr\_key\_value}}{\emph{attr}, \emph{key}}{}
\end{fulllineitems}

\index{get\_attr\_list() (zBuilder.nodes.base.BaseNode method)}

\begin{fulllineitems}
\phantomsection\label{zBuilder.nodes:zBuilder.nodes.base.BaseNode.get_attr_list}\pysiglinewithargsret{\bfcode{get\_attr\_list}}{}{}
gets list of attribute names stored with node
\begin{quote}\begin{description}
\item[{Returns}] \leavevmode
{[}{]} of attribute names

\end{description}\end{quote}

\end{fulllineitems}

\index{get\_attr\_value() (zBuilder.nodes.base.BaseNode method)}

\begin{fulllineitems}
\phantomsection\label{zBuilder.nodes:zBuilder.nodes.base.BaseNode.get_attr_value}\pysiglinewithargsret{\bfcode{get\_attr\_value}}{\emph{attr}}{}
gets value of an attribute in node
\begin{quote}\begin{description}
\item[{Parameters}] \leavevmode
\textbf{\texttt{attr}} (\emph{str}) -- The attribute to get value of

\item[{Returns}] \leavevmode
value of attribute

\end{description}\end{quote}

\end{fulllineitems}

\index{get\_maps() (zBuilder.nodes.base.BaseNode method)}

\begin{fulllineitems}
\phantomsection\label{zBuilder.nodes:zBuilder.nodes.base.BaseNode.get_maps}\pysiglinewithargsret{\bfcode{get\_maps}}{}{}
\end{fulllineitems}

\index{get\_name() (zBuilder.nodes.base.BaseNode method)}

\begin{fulllineitems}
\phantomsection\label{zBuilder.nodes:zBuilder.nodes.base.BaseNode.get_name}\pysiglinewithargsret{\bfcode{get\_name}}{\emph{longName=False}}{}
get name of node
\begin{quote}\begin{description}
\item[{Parameters}] \leavevmode
\textbf{\texttt{longName}} (\emph{bool}) -- If True returns the long name of node.  Defaults to \textbf{False}

\item[{Returns}] \leavevmode
(str) of node name

\end{description}\end{quote}

\end{fulllineitems}

\index{get\_type() (zBuilder.nodes.base.BaseNode method)}

\begin{fulllineitems}
\phantomsection\label{zBuilder.nodes:zBuilder.nodes.base.BaseNode.get_type}\pysiglinewithargsret{\bfcode{get\_type}}{}{}
get type of node
\begin{quote}\begin{description}
\item[{Returns}] \leavevmode
(str) of node name

\end{description}\end{quote}

\end{fulllineitems}

\index{print\_() (zBuilder.nodes.base.BaseNode method)}

\begin{fulllineitems}
\phantomsection\label{zBuilder.nodes:zBuilder.nodes.base.BaseNode.print_}\pysiglinewithargsret{\bfcode{print\_}}{}{}
\end{fulllineitems}

\index{set\_association() (zBuilder.nodes.base.BaseNode method)}

\begin{fulllineitems}
\phantomsection\label{zBuilder.nodes:zBuilder.nodes.base.BaseNode.set_association}\pysiglinewithargsret{\bfcode{set\_association}}{\emph{association}}{}
\end{fulllineitems}

\index{set\_attr\_key\_value() (zBuilder.nodes.base.BaseNode method)}

\begin{fulllineitems}
\phantomsection\label{zBuilder.nodes:zBuilder.nodes.base.BaseNode.set_attr_key_value}\pysiglinewithargsret{\bfcode{set\_attr\_key\_value}}{\emph{attr}, \emph{key}, \emph{value}}{}
\end{fulllineitems}

\index{set\_attr\_value() (zBuilder.nodes.base.BaseNode method)}

\begin{fulllineitems}
\phantomsection\label{zBuilder.nodes:zBuilder.nodes.base.BaseNode.set_attr_value}\pysiglinewithargsret{\bfcode{set\_attr\_value}}{\emph{attr}, \emph{value}}{}
sets value of an attribute in node
\begin{quote}\begin{description}
\item[{Parameters}] \leavevmode\begin{itemize}
\item {} 
\textbf{\texttt{attr}} (\emph{str}) -- The attribute to get value of

\item {} 
\textbf{\texttt{value}} -- the value to set

\end{itemize}

\end{description}\end{quote}

\end{fulllineitems}

\index{set\_attrs() (zBuilder.nodes.base.BaseNode method)}

\begin{fulllineitems}
\phantomsection\label{zBuilder.nodes:zBuilder.nodes.base.BaseNode.set_attrs}\pysiglinewithargsret{\bfcode{set\_attrs}}{\emph{attrs}}{}
\end{fulllineitems}

\index{set\_maps() (zBuilder.nodes.base.BaseNode method)}

\begin{fulllineitems}
\phantomsection\label{zBuilder.nodes:zBuilder.nodes.base.BaseNode.set_maps}\pysiglinewithargsret{\bfcode{set\_maps}}{\emph{maps}}{}
\end{fulllineitems}

\index{set\_name() (zBuilder.nodes.base.BaseNode method)}

\begin{fulllineitems}
\phantomsection\label{zBuilder.nodes:zBuilder.nodes.base.BaseNode.set_name}\pysiglinewithargsret{\bfcode{set\_name}}{\emph{name}}{}
Sets name of node
\begin{quote}\begin{description}
\item[{Parameters}] \leavevmode
\textbf{\texttt{name}} (\emph{str}) -- the name of node.

\end{description}\end{quote}

\end{fulllineitems}

\index{set\_type() (zBuilder.nodes.base.BaseNode method)}

\begin{fulllineitems}
\phantomsection\label{zBuilder.nodes:zBuilder.nodes.base.BaseNode.set_type}\pysiglinewithargsret{\bfcode{set\_type}}{\emph{type\_}}{}
Sets type of node
\begin{quote}\begin{description}
\item[{Parameters}] \leavevmode
\textbf{\texttt{type}} (\emph{str}) -- the type of node.

\end{description}\end{quote}

\end{fulllineitems}

\index{string\_replace() (zBuilder.nodes.base.BaseNode method)}

\begin{fulllineitems}
\phantomsection\label{zBuilder.nodes:zBuilder.nodes.base.BaseNode.string_replace}\pysiglinewithargsret{\bfcode{string\_replace}}{\emph{search}, \emph{replace}}{}
\end{fulllineitems}


\end{fulllineitems}

\index{build\_attr\_key\_values() (in module zBuilder.nodes.base)}

\begin{fulllineitems}
\phantomsection\label{zBuilder.nodes:zBuilder.nodes.base.build_attr_key_values}\pysiglinewithargsret{\code{zBuilder.nodes.base.}\bfcode{build\_attr\_key\_values}}{\emph{selection}, \emph{attrList}}{}
\end{fulllineitems}

\index{build\_attr\_list() (in module zBuilder.nodes.base)}

\begin{fulllineitems}
\phantomsection\label{zBuilder.nodes:zBuilder.nodes.base.build_attr_list}\pysiglinewithargsret{\code{zBuilder.nodes.base.}\bfcode{build\_attr\_list}}{\emph{selection}, \emph{attr\_filter=None}}{}
\end{fulllineitems}

\index{replace\_longname() (in module zBuilder.nodes.base)}

\begin{fulllineitems}
\phantomsection\label{zBuilder.nodes:zBuilder.nodes.base.replace_longname}\pysiglinewithargsret{\code{zBuilder.nodes.base.}\bfcode{replace\_longname}}{\emph{search}, \emph{replace}, \emph{longName}}{}
does a search and replace on a long name.  It splits it up by (`\textbar{}') then
performs it on each piece
\begin{quote}\begin{description}
\item[{Parameters}] \leavevmode\begin{itemize}
\item {} 
\textbf{\texttt{search}} (\emph{str}) -- search term

\item {} 
\textbf{\texttt{replace}} (\emph{str}) -- replace term

\item {} 
\textbf{\texttt{longName}} (\emph{str}) -- the long name to perform action on

\end{itemize}

\item[{Returns}] \leavevmode
result of search and replace

\item[{Return type}] \leavevmode
str

\end{description}\end{quote}

\end{fulllineitems}

\index{set\_attrs() (in module zBuilder.nodes.base)}

\begin{fulllineitems}
\phantomsection\label{zBuilder.nodes:zBuilder.nodes.base.set_attrs}\pysiglinewithargsret{\code{zBuilder.nodes.base.}\bfcode{set\_attrs}}{\emph{nodes}, \emph{attr\_filter=None}}{}
\end{fulllineitems}



\subsubsection{zBuilder.nodes.zEmbedder}
\label{zBuilder.nodes:module-zBuilder.nodes.zEmbedder}\label{zBuilder.nodes:zbuilder-nodes-zembedder}\index{zBuilder.nodes.zEmbedder (module)}\index{EmbedderNode (class in zBuilder.nodes.zEmbedder)}

\begin{fulllineitems}
\phantomsection\label{zBuilder.nodes:zBuilder.nodes.zEmbedder.EmbedderNode}\pysigline{\strong{class }\code{zBuilder.nodes.zEmbedder.}\bfcode{EmbedderNode}}
Bases: {\hyperref[zBuilder.nodes:zBuilder.nodes.base.BaseNode]{\emph{\code{zBuilder.nodes.base.BaseNode}}}}
\index{get\_collision\_meshes() (zBuilder.nodes.zEmbedder.EmbedderNode method)}

\begin{fulllineitems}
\phantomsection\label{zBuilder.nodes:zBuilder.nodes.zEmbedder.EmbedderNode.get_collision_meshes}\pysiglinewithargsret{\bfcode{get\_collision\_meshes}}{\emph{longName=False}}{}
\end{fulllineitems}

\index{get\_embedded\_meshes() (zBuilder.nodes.zEmbedder.EmbedderNode method)}

\begin{fulllineitems}
\phantomsection\label{zBuilder.nodes:zBuilder.nodes.zEmbedder.EmbedderNode.get_embedded_meshes}\pysiglinewithargsret{\bfcode{get\_embedded\_meshes}}{\emph{longName=False}}{}
\end{fulllineitems}

\index{set\_collision\_meshes() (zBuilder.nodes.zEmbedder.EmbedderNode method)}

\begin{fulllineitems}
\phantomsection\label{zBuilder.nodes:zBuilder.nodes.zEmbedder.EmbedderNode.set_collision_meshes}\pysiglinewithargsret{\bfcode{set\_collision\_meshes}}{\emph{meshes}}{}
\end{fulllineitems}

\index{set\_embedded\_meshes() (zBuilder.nodes.zEmbedder.EmbedderNode method)}

\begin{fulllineitems}
\phantomsection\label{zBuilder.nodes:zBuilder.nodes.zEmbedder.EmbedderNode.set_embedded_meshes}\pysiglinewithargsret{\bfcode{set\_embedded\_meshes}}{\emph{meshes}}{}
\end{fulllineitems}

\index{string\_replace() (zBuilder.nodes.zEmbedder.EmbedderNode method)}

\begin{fulllineitems}
\phantomsection\label{zBuilder.nodes:zBuilder.nodes.zEmbedder.EmbedderNode.string_replace}\pysiglinewithargsret{\bfcode{string\_replace}}{\emph{search}, \emph{replace}, \emph{name=True}, \emph{association=True}}{}
\end{fulllineitems}


\end{fulllineitems}

\index{get\_embedded\_meshes() (in module zBuilder.nodes.zEmbedder)}

\begin{fulllineitems}
\phantomsection\label{zBuilder.nodes:zBuilder.nodes.zEmbedder.get_embedded_meshes}\pysiglinewithargsret{\code{zBuilder.nodes.zEmbedder.}\bfcode{get\_embedded\_meshes}}{\emph{bodies}}{}
\end{fulllineitems}

\index{get\_zEmbedder() (in module zBuilder.nodes.zEmbedder)}

\begin{fulllineitems}
\phantomsection\label{zBuilder.nodes:zBuilder.nodes.zEmbedder.get_zEmbedder}\pysiglinewithargsret{\code{zBuilder.nodes.zEmbedder.}\bfcode{get\_zEmbedder}}{\emph{bodies}}{}
\end{fulllineitems}

\index{get\_zGeos() (in module zBuilder.nodes.zEmbedder)}

\begin{fulllineitems}
\phantomsection\label{zBuilder.nodes:zBuilder.nodes.zEmbedder.get_zGeos}\pysiglinewithargsret{\code{zBuilder.nodes.zEmbedder.}\bfcode{get\_zGeos}}{\emph{bodies}}{}
\end{fulllineitems}

\index{replace\_longname() (in module zBuilder.nodes.zEmbedder)}

\begin{fulllineitems}
\phantomsection\label{zBuilder.nodes:zBuilder.nodes.zEmbedder.replace_longname}\pysiglinewithargsret{\code{zBuilder.nodes.zEmbedder.}\bfcode{replace\_longname}}{\emph{search}, \emph{replace}, \emph{longName}}{}
does a search and replace on a long name.  It splits it up by (`\textbar{}') then
performs it on each piece
\begin{quote}\begin{description}
\item[{Parameters}] \leavevmode\begin{itemize}
\item {} 
\textbf{\texttt{search}} (\emph{str}) -- search term

\item {} 
\textbf{\texttt{replace}} (\emph{str}) -- replace term

\item {} 
\textbf{\texttt{longName}} (\emph{str}) -- the long name to perform action on

\end{itemize}

\item[{Returns}] \leavevmode
result of search and replace

\item[{Return type}] \leavevmode
str

\end{description}\end{quote}

\end{fulllineitems}



\subsubsection{zBuilder.nodes.zTet}
\label{zBuilder.nodes:module-zBuilder.nodes.zTet}\label{zBuilder.nodes:zbuilder-nodes-ztet}\index{zBuilder.nodes.zTet (module)}\index{TetNode (class in zBuilder.nodes.zTet)}

\begin{fulllineitems}
\phantomsection\label{zBuilder.nodes:zBuilder.nodes.zTet.TetNode}\pysigline{\strong{class }\code{zBuilder.nodes.zTet.}\bfcode{TetNode}}
Bases: {\hyperref[zBuilder.nodes:zBuilder.nodes.base.BaseNode]{\emph{\code{zBuilder.nodes.base.BaseNode}}}}
\index{get\_user\_tet\_mesh() (zBuilder.nodes.zTet.TetNode method)}

\begin{fulllineitems}
\phantomsection\label{zBuilder.nodes:zBuilder.nodes.zTet.TetNode.get_user_tet_mesh}\pysiglinewithargsret{\bfcode{get\_user\_tet\_mesh}}{\emph{longName=False}}{}
\end{fulllineitems}

\index{print\_() (zBuilder.nodes.zTet.TetNode method)}

\begin{fulllineitems}
\phantomsection\label{zBuilder.nodes:zBuilder.nodes.zTet.TetNode.print_}\pysiglinewithargsret{\bfcode{print\_}}{}{}
\end{fulllineitems}

\index{set\_user\_tet\_mesh() (zBuilder.nodes.zTet.TetNode method)}

\begin{fulllineitems}
\phantomsection\label{zBuilder.nodes:zBuilder.nodes.zTet.TetNode.set_user_tet_mesh}\pysiglinewithargsret{\bfcode{set\_user\_tet\_mesh}}{\emph{mesh}}{}
\end{fulllineitems}

\index{string\_replace() (zBuilder.nodes.zTet.TetNode method)}

\begin{fulllineitems}
\phantomsection\label{zBuilder.nodes:zBuilder.nodes.zTet.TetNode.string_replace}\pysiglinewithargsret{\bfcode{string\_replace}}{\emph{search}, \emph{replace}}{}
\end{fulllineitems}


\end{fulllineitems}

\index{replace\_longname() (in module zBuilder.nodes.zTet)}

\begin{fulllineitems}
\phantomsection\label{zBuilder.nodes:zBuilder.nodes.zTet.replace_longname}\pysiglinewithargsret{\code{zBuilder.nodes.zTet.}\bfcode{replace\_longname}}{\emph{search}, \emph{replace}, \emph{longName}}{}
does a search and replace on a long name.  It splits it up by (`\textbar{}') then
performs it on each piece
\begin{quote}\begin{description}
\item[{Parameters}] \leavevmode\begin{itemize}
\item {} 
\textbf{\texttt{search}} (\emph{str}) -- search term

\item {} 
\textbf{\texttt{replace}} (\emph{str}) -- replace term

\item {} 
\textbf{\texttt{longName}} (\emph{str}) -- the long name to perform action on

\end{itemize}

\item[{Returns}] \leavevmode
result of search and replace

\item[{Return type}] \leavevmode
str

\end{description}\end{quote}

\end{fulllineitems}



\section{zBuilder.zMaya}
\label{zBuilder:zbuilder-zmaya}\label{zBuilder:module-zBuilder.zMaya}\index{zBuilder.zMaya (module)}\index{clean\_scene() (in module zBuilder.zMaya)}

\begin{fulllineitems}
\phantomsection\label{zBuilder:zBuilder.zMaya.clean_scene}\pysiglinewithargsret{\code{zBuilder.zMaya.}\bfcode{clean\_scene}}{}{}
\end{fulllineitems}

\index{create\_zBone() (in module zBuilder.zMaya)}

\begin{fulllineitems}
\phantomsection\label{zBuilder:zBuilder.zMaya.create_zBone}\pysiglinewithargsret{\code{zBuilder.zMaya.}\bfcode{create\_zBone}}{\emph{bodies}}{}
\end{fulllineitems}

\index{getDependNode() (in module zBuilder.zMaya)}

\begin{fulllineitems}
\phantomsection\label{zBuilder:zBuilder.zMaya.getDependNode}\pysiglinewithargsret{\code{zBuilder.zMaya.}\bfcode{getDependNode}}{\emph{nodeName}}{}
Get an MObject (depend node) for the associated node name
\begin{quote}\begin{description}
\item[{Parameters}] \leavevmode
\textbf{\texttt{nodeName}} -- String representing the node

\item[{Return}] \leavevmode
depend node (MObject)

\end{description}\end{quote}

\end{fulllineitems}

\index{get\_association() (in module zBuilder.zMaya)}

\begin{fulllineitems}
\phantomsection\label{zBuilder:zBuilder.zMaya.get_association}\pysiglinewithargsret{\code{zBuilder.zMaya.}\bfcode{get\_association}}{\emph{zNode}}{}
\end{fulllineitems}

\index{get\_type() (in module zBuilder.zMaya)}

\begin{fulllineitems}
\phantomsection\label{zBuilder:zBuilder.zMaya.get_type}\pysiglinewithargsret{\code{zBuilder.zMaya.}\bfcode{get\_type}}{\emph{body}}{}
\end{fulllineitems}

\index{get\_zAttachments() (in module zBuilder.zMaya)}

\begin{fulllineitems}
\phantomsection\label{zBuilder:zBuilder.zMaya.get_zAttachments}\pysiglinewithargsret{\code{zBuilder.zMaya.}\bfcode{get\_zAttachments}}{\emph{bodies}}{}
\end{fulllineitems}

\index{get\_zBones() (in module zBuilder.zMaya)}

\begin{fulllineitems}
\phantomsection\label{zBuilder:zBuilder.zMaya.get_zBones}\pysiglinewithargsret{\code{zBuilder.zMaya.}\bfcode{get\_zBones}}{\emph{bodies}}{}
\end{fulllineitems}

\index{get\_zFibers() (in module zBuilder.zMaya)}

\begin{fulllineitems}
\phantomsection\label{zBuilder:zBuilder.zMaya.get_zFibers}\pysiglinewithargsret{\code{zBuilder.zMaya.}\bfcode{get\_zFibers}}{\emph{bodies}}{}
\end{fulllineitems}

\index{get\_zMaterials() (in module zBuilder.zMaya)}

\begin{fulllineitems}
\phantomsection\label{zBuilder:zBuilder.zMaya.get_zMaterials}\pysiglinewithargsret{\code{zBuilder.zMaya.}\bfcode{get\_zMaterials}}{\emph{bodies}}{}
Gets zMaterial nodes given a mesh

\end{fulllineitems}

\index{get\_zSolver() (in module zBuilder.zMaya)}

\begin{fulllineitems}
\phantomsection\label{zBuilder:zBuilder.zMaya.get_zSolver}\pysiglinewithargsret{\code{zBuilder.zMaya.}\bfcode{get\_zSolver}}{\emph{body}}{}
\end{fulllineitems}

\index{get\_zSolverTransform() (in module zBuilder.zMaya)}

\begin{fulllineitems}
\phantomsection\label{zBuilder:zBuilder.zMaya.get_zSolverTransform}\pysiglinewithargsret{\code{zBuilder.zMaya.}\bfcode{get\_zSolverTransform}}{\emph{body}}{}
\end{fulllineitems}

\index{get\_zTet\_user\_mesh() (in module zBuilder.zMaya)}

\begin{fulllineitems}
\phantomsection\label{zBuilder:zBuilder.zMaya.get_zTet_user_mesh}\pysiglinewithargsret{\code{zBuilder.zMaya.}\bfcode{get\_zTet\_user\_mesh}}{\emph{zNode}}{}
\end{fulllineitems}

\index{get\_zTets() (in module zBuilder.zMaya)}

\begin{fulllineitems}
\phantomsection\label{zBuilder:zBuilder.zMaya.get_zTets}\pysiglinewithargsret{\code{zBuilder.zMaya.}\bfcode{get\_zTets}}{\emph{bodies}}{}
\end{fulllineitems}

\index{get\_zTissues() (in module zBuilder.zMaya)}

\begin{fulllineitems}
\phantomsection\label{zBuilder:zBuilder.zMaya.get_zTissues}\pysiglinewithargsret{\code{zBuilder.zMaya.}\bfcode{get\_zTissues}}{\emph{bodies}}{}
\end{fulllineitems}

\index{isSolver() (in module zBuilder.zMaya)}

\begin{fulllineitems}
\phantomsection\label{zBuilder:zBuilder.zMaya.isSolver}\pysiglinewithargsret{\code{zBuilder.zMaya.}\bfcode{isSolver}}{\emph{selection}}{}
\end{fulllineitems}

\index{rename\_ziva\_nodes() (in module zBuilder.zMaya)}

\begin{fulllineitems}
\phantomsection\label{zBuilder:zBuilder.zMaya.rename_ziva_nodes}\pysiglinewithargsret{\code{zBuilder.zMaya.}\bfcode{rename\_ziva\_nodes}}{}{}
\end{fulllineitems}

\index{select\_tissue\_meshes() (in module zBuilder.zMaya)}

\begin{fulllineitems}
\phantomsection\label{zBuilder:zBuilder.zMaya.select_tissue_meshes}\pysiglinewithargsret{\code{zBuilder.zMaya.}\bfcode{select\_tissue\_meshes}}{}{}
\end{fulllineitems}



\chapter{Indices and tables}
\label{index:indices-and-tables}\begin{itemize}
\item {} 
\DUspan{xref,std,std-ref}{genindex}

\item {} 
\DUspan{xref,std,std-ref}{modindex}

\item {} 
\DUspan{xref,std,std-ref}{search}

\end{itemize}


\renewcommand{\indexname}{Python Module Index}
\begin{theindex}
\def\bigletter#1{{\Large\sffamily#1}\nopagebreak\vspace{1mm}}
\bigletter{z}
\item {\texttt{zBuilder.data}}, \pageref{zBuilder.data:module-zBuilder.data}
\item {\texttt{zBuilder.data.map}}, \pageref{zBuilder.data:module-zBuilder.data.map}
\item {\texttt{zBuilder.data.mesh}}, \pageref{zBuilder.data:module-zBuilder.data.mesh}
\item {\texttt{zBuilder.nodeCollection}}, \pageref{zBuilder:module-zBuilder.nodeCollection}
\item {\texttt{zBuilder.nodes.base}}, \pageref{zBuilder.nodes:module-zBuilder.nodes.base}
\item {\texttt{zBuilder.nodes.zEmbedder}}, \pageref{zBuilder.nodes:module-zBuilder.nodes.zEmbedder}
\item {\texttt{zBuilder.nodes.zTet}}, \pageref{zBuilder.nodes:module-zBuilder.nodes.zTet}
\item {\texttt{zBuilder.setup.Attributes}}, \pageref{zBuilder.setup:module-zBuilder.setup.Attributes}
\item {\texttt{zBuilder.setup.Constraints}}, \pageref{zBuilder.setup:module-zBuilder.setup.Constraints}
\item {\texttt{zBuilder.setup.Selection}}, \pageref{zBuilder.setup:module-zBuilder.setup.Selection}
\item {\texttt{zBuilder.setup.Ziva}}, \pageref{zBuilder.setup:module-zBuilder.setup.Ziva}
\item {\texttt{zBuilder.zMaya}}, \pageref{zBuilder:module-zBuilder.zMaya}
\end{theindex}

\renewcommand{\indexname}{Index}
\printindex
\end{document}
